\documentclass[12pt]{article}
\usepackage{amsmath}
\usepackage{amssymb}
\usepackage{amsthm}
\usepackage{tikz}
\usepackage{parskip}
\renewcommand{\qedsymbol}{\rule{0.7em}{0.7em}}

\title{Solutions to Problems in Morton Hamermesh's ``Group Theory and its
Application to Physical Problems", 2nd Ed. (Dover, 1989)}
\author{William Gertler}

\begin{document}
\maketitle

Something that bothers me about the undergraduate physics curriculum at
Waterloo is that despite the wide array of quantum theory courses you can
(and indeed, must) take, students are not required to take a single course in
abstract algebra. Even those in mathematical physics needn't so much as look at
the pure math calendar to get their degree.

Modern quantum mechanical theory relies very heavily on the
language of groups and algebras \textendash terms that only increase in their importance
the higher up the chain you go. You hear things like ``Poisson bracket" and 
``Lie group" and perhaps even ``Lie Algebra". You hear about Noether's theorem,
but only her work on variational calculus. If you take particle physics,
you may hear about special unitary groups and that sort of thing.

It hardly suffices to take these terms for granted if one wants to dive into the
weeds of modern-day mathematical physics, and there are few places more worthy
of building foundational knowledge than in group theory. So, hopefully you try
the problems out before looking at my solutions here. That's the only way to
really learn, after all. But this will be here should you get truly stuck, and
no mathematician is available to consult.

Hamermesh's text does not number the problems, but the sections are broken up
nicely enough that they're not hard to find, provided I index those sections
here. I will also write out the questions themselves so that searching for them
will be even more simple.

\newpage

\textbf{CHAPTER 1: ELEMENTS OF GROUP THEORY}

\textbf{1-1 Correspondences and Transformations.}

\underline{Problem 1-1.1:} The cross ratio of four points on a line is defined
as
\[ 
\frac{(x_1 - x_2)/(x_3 - x_2)}{(x_1-x_4)/(x_3-x_4)} =
\frac{(x_1 - x_2)(x_3 - x_4)}{(x_1-x_4)(x_3-x_2)},
\]
where $x_1$, $x_2$, $x_3$, $x_4$ are the coordinates of the four points. Show
that the ratio is \textit{invariant} under projective transformation, i.e. that
the cross ratio obtained from the image point has the same form as that for the
object points.

\underline{Solution:} All that needs to be done is verify that the image of x
under a projective transformation results in the same equation as that stated
in the question. The projection of these points gives
\[
    x'_i = \frac{ax_i+b}{cx_i+d}, \text{ where } ad-bc \neq 0.
\]
for $i \in {1,2,3,4}$. By direct calculation, we can see that
\begin{align*}
    & \frac{\left(\frac{ax_1 + b}{cx_1 + d} - \frac{ax_2 + b}{cx_2 + d}\right)
    \left( \frac{ax_3 + b}{cx_3 + d} - \frac{ax_4 + b}{cx_4 + d}\right)}
    {\left(\frac{ax_1 + b}{cx_1 + d} - \frac{ax_4 + b}{cx_4 + d}\right)    
    \left( \frac{ax_3 + b}{cx_3 + d} - \frac{ax_2 + b}{cx_2 + d}\right)}\\
    \\
    &=
    \frac{((cx_2+d)(ax_1+b)-(cx_1+d)(ax_2+b))((cx_4+d)(ax_3+b)-(cx_3+d)(ax_2+b))}
      {((cx_4+d)(ax_1+b)-(cx_1+d)(ax_4+b))((cx_2+d)(ax_3+b)-(cx_3+d)(ax_2+b))}
\end{align*}
after some tedious algebra. After a little more computing, you can simplify
this statement to
\begin{align*}
    & \frac{((ad-bc)(x_1-x_2))((ad-bc)(x_3-x_4))}{((ad-bc)(x_1-x_4))((ad-bc)(x_3-x_2))}\\
    \\
    & = \frac{(x_1 - x_2)(x_3 - x_4)}{(x_1-x_4)(x_3-x_2)}\\
    \\
    & = \frac{(x_1 - x_2)/(x_3 - x_2)}{(x_1-x_4)/(x_3-x_4)}
\end{align*}
which is the same as the cross ratio between object points. Thus, the cross
ratio of a line is an invariant under projective transformations of that line.

\newpage

\textbf{1-2 Groups. Definition and Examples.}

\underline{Problem 1-2.1:} Show that the cyclic group of order 4 and the Klein
4-group are the only possible structures for groups of order 4.

\underline{Solution:} This problem would be a very quick exercise if at this
point the reader was familiar with Lagrange's theorem, but that hasn't been
covered in the text at this point so I will assume the reader doesn't know this
one. It's still solvable, but it takes more effort (and paper).

Groups need to fulfill certain conditions, which may be reflected in a Cayley
table. In particular, each element in a column (and likewise, a row) may only
appear once per column and per row. Further, each element of the group must
appear in each column and each row. These greatly reduce the number of binary
structures that could represent groups from $4^{16}$ down to 4. We can write
those here:

\begin{tabular}{c | c c c c}
    A & e & a & b & c \\
    \cline{1-5}
    e & e & a & b & c \\
    a & a & e & c & b \\
    b & b & c & a & e \\
    c & c & b & e & a \\
\end{tabular} \quad , \quad 
\begin{tabular}{c | c c c c}
    B & e & a & b & c \\
    \cline{1-5}
    e & e & a & b & c \\
    a & a & b & c & e \\
    b & b & c & e & a \\
    c & c & e & a & b \\
\end{tabular} 

\begin{tabular}{c | c c c c}
    C & e & a & b & c \\
    \cline{1-5}
    e & e & a & b & c \\
    a & a & c & e & b \\
    b & b & e & c & a \\
    c & c & b & a & e \\
\end{tabular} \quad , \quad 
\noindent\begin{tabular}{c | c c c c}
    D & e & a & b & c \\
    \cline{1-5}
    e & e & a & b & c \\
    a & a & e & c & b \\
    b & b & c & e & a \\
    c & c & b & a & e \\
\end{tabular}.

Now, it turns out that the the tables $A$, $B$, $C$ are identical up to an
isomorphism. They are

\begin{align*}
    &\phi_1(A) = \begin{cases}
                \phi(e) &= e\\
                \phi(a) &= b\\
                \phi(b) &= a\\
                \phi(c) &= c\\
                \end{cases} \quad \quad 
    &\phi_2(A) = \begin{cases}
                \phi(e) &= e\\
                \phi(a) &= c\\
                \phi(b) &= b\\
                \phi(c) &= a\\
                \end{cases}
\end{align*}
which take $A \rightarrow B$ and $A \rightarrow C$, respectively. We recognize
these three as one group, which is the cyclic 4-group. Existence and uniqueness
of this group for any positive integer $n$ is clear from modular arithmetic
over the naturals. $D$ is the Klein 4-Group.

\underline{Problem 1-2.2:} Show that the group of order 4 is Abelian.

\underline{Solution:} After those two, this one is very quick. We simply
observe that the Cayley tables for the order 4 groups are all symmetrical.
Thus, the groups they represent are Abelian.

\underline{Problem 1-2.3:} Give a realization of each group of order 4.

\underline{Solution:} An example of the cyclic 4-group is
$\langle \mathbb{Z}_4 , +_4 \rangle$. An example of the Klein 4-group is
$\langle \mathbb{Z}_2 \times \mathbb{Z}_2 , +_2 \rangle $.

\newpage

\textbf{1-3 Subgroups. Cayley's Theorem.}

\underline{Problem 1-3.1:} Give the elements of the regular subgroup of $S_6$
which is isomorphic with the cyclic group of order 6.

\underline{Solution:} To see the regular subgroup isomorphic to $C_6$, we need
only write one of the cayley tables for that group and read off the results:
\begin{align*}
    \noindent\begin{tabular}{c| c c c c c c}
        & e & a & b & c & d & f\\
       \cline{1-6}
       e & e & a & b & c & d & f\\
       a & a & b & c & d & f & e\\
       b & b & c & d & f & e & a\\
       c & c & d & f & e & a & b\\
       d & d & f & e & a & b & c\\
       f & f & e & a & b & c & d\\ 
    \end{tabular} \rightarrow 
    \begin{cases}
        \binom{123456}{123456} &= e\\
        \binom{123456}{234561} &= (123456)\\
        \binom{123456}{345612} &= (135)(246)\\
        \binom{123456}{456123} &= (14)(25)(36)\\
        \binom{123456}{561234} &= (153)(264)\\
        \binom{123456}{612345} &= (165432)\\
    \end{cases}.
\end{align*}
We can see that this group is indeed regular. These are all the elements, up to
some isomorphisms.

\underline{Problem 1-3.2:} Use Cayley's theorem to find the possible structures
of groups of order 6.

\underline{Solution:} Cayley's theorem states that every group $G$ of order $n$
is isomorphic with a subgroup of the symmetric group $S_n$. In our case, $n=6$.
From our work above, we have already done much of our task. We can further break
down our result: we presume there are no allowed 6-cycles. Then, because the
cycles need be of the same length, we can have two 3-cycles and three 2-cycles,
plus $e$. That gives us
\[ 
\{(123), (132), (12), (13), (23), e\},
\]
which happens to be the symmetric 3-group, $S_3$.

Is that it? It is! How do we know there aren't more? Well, suppose we disallow
3-cycles. Then $G$ has only 2-cycles and the identity. The group has to have 6
elements in total, so we need to have 5 2-cycles as well as the identity. But
by the theorem, that puts us at trying to find a group isomorphic to a subset
of $S_{10}$, not $S_{6}$. It's just too big.

So there are only two ways to structure a group of order 6: the Abelian group
$C_6$ and the non-Abelian group $S_3$.

\newpage

\textbf{1-4 Cosets. Lagrange's Theorem.}

\underline{Problem 1-4.1:} The cyclic permutations of four symbols form a
subgroup $\mathcal{H}$ of $S_4$. Resolve $S_4$ into left cosets with respect to
$\mathcal{H}$. Compare this resolution with one into right cosets.

\underline{Solution:} Some of the work has been done for us by this point in
the text. We need only look to the previous section to see that the cyclic
group of order 4 is given by

\[ 
C_4 = \{ e, (1234), (13)(24), (1432)\}.
\]

To break up $S_4$ into cosets partitioned by $C_4$, we need only apply elements
of $S_4$ that are not present in $C_4$. Doing this successively and checking
to make sure none of your needed elements have already been used, you can
construct $S_4$ in a straightforward manner. 

I will write the solution I found and give an example for comparison with the
right cosets afterwards: we find that 

\[
    S_4 = C_4 + (12)C_4 + (13)C_4 + (14)C_4 + (23)C_4 + (34)C_4.
\]

Nice. Let's compare an example of left and right cosets.

\[
    (12)C_4 = \{ (12), (234), (1324), 143)  \}
\]

is the left coset. Our right coset is

\[
    C_4(12) = \{ (12), (134), (1423), (243)  \}.
\]

They're not the same! And in fact, they generally will not be the same ---
that's because the application of permutations may lead to differing and 
dependent cycles, which do not commute. Those portions partition portions that
are equal for both the left and right coset are called "normal subgroups",
and will be covered later in the text. In generality the left and right cosets
will have the same index and produce the same results in summation. 

\underline{Problem 1-4.2:} Find the possible structures of groups of order 8.

\underline{Solution:} By Lagrange's theorem, we know that the elements of the
group must have order dividing the order of the group. For a group of order 8,
that means the elements may take orders 1,2,4, and 8. 

If the group has an order-8 element, it has to be the cyclic group of order 8,
$C_8$. 

Next, we treat the case in which there are neither order-8 or order-4 elements.
In this case, every element is order-2. Put in another way, this means each
element is its own inverse. We shall take a moment here to prove an important
property of groups with such a structure of any numver of elements:

\textbf{Theorem:} Any group $G$ where the order of each nontrivial element
$a \in G$ is 2 must be abelian. 

\begin{proof}
suppose $a,b \in G$. Then
\[
    a \cdot b = (b \cdot b) \cdot a \cdot b \cdot (a \cdot a).
\]
\end{proof}
By group axioms, we know that the operation ``$\cdot$" is associative, and we
know that the group is closed --- therefore, $a \cdot b$ is also an order-2
element by the supposition of this theorem. With that in mind, we see
\begin{align*}
    a \cdot b &= (b \cdot b) \cdot a \cdot b \cdot (a \cdot a)\\
              &= b \cdot (b \cdot a) \cdot (b \cdot a) \cdot a\\
              &= b \cdot a.
\end{align*}
Ergo, $G$ is an abelian group. Now, for unique elements $a,b \in G$ we can add
a third unique element so that we achieve an order-8 group. We call this
element $c$ and write
\[
    G = \{ e, a, b, c, ab, ac, bc, abc\}.
\]
Even better, we may write that the group is generated in the following way:
\[
 G = \langle a,b,c \ |\  a^2 = b^2 = c^2 = e, ab = ba, ac = ca, bc = cb \rangle. 
\]
From here, we can see that this group is isomorphic to
$\mathbb{Z}_2 \times \mathbb{Z}_2 \times \mathbb{Z}_2$, under the
transformation
$f:G \rightarrow \mathbb{Z}_2 \times \mathbb{Z}_2 \times \mathbb{Z}_2$ where
\[ 
f(a) = (1,0,0), \ f(b) = (0,1,0), \ f(c) = (0,0,1)
\]
under combination rules of addition modulo 2. 

Now lets suppose that $G$ does in fact have an element of order 4, which we
will call $x$. Then we have a subgroup $H \subset G = \{ e, x, x^2, x^3\}$.
We can decompose $G$ into cosets, say then that $y$ generates that coset:
\begin{align*}
    G  &= H + yH, \\
    yH &= \{ y, yx, yx^2, yx^3\}.
\end{align*}

Great. Now what kind of elements are in the coset generated by $y$? Well, let's
assume that all the elements in $yH$ are of order 4 and see what happens. If
that is the case, then the only element of order-2 in G is $x^2$. However ---
if we square any element in the coset, because each is order 4, the result must
be order 2. As $x^2$ is the only order-2 element in G, any squared element of
the coset $yH$ must be equal to $x^2$. We're coming up on a problem:
\[
    (yx)^2 = y^2x^2 = x^2 \rightarrow y^2 = e.
\]
There's no way for every element of $yH$ to be order-4. We need at least one
element of order-2, which we will call $y'$. We can use that element to
generate our coset instead, and so
\[
    G = H + y'H= \{ e, x, x^2, x^3, y', y'x, y'x^2, y'x^3\}.
\]
We can see that this is generated by a relation
$G =\{y^{\prime i}x^j\},\ (i,j) \in \mathbb{Z}_2 \times \mathbb{Z}_4$,
so $G$ in this case is isomorphic to $\mathbb{Z}_2 \times \mathbb{Z}_4$.

That's it for the abelian cases. We cannot have all elements of order-1 or
order-2, and we cannot have an element of order-8. So our remaining cases must
contain order-4 elements. Let $x$ be that order-4 element, which again
generates a subgroup $H \subset G = \{ e, x, x^2, x^3 \}$. This subgroup of $G$
gives an index of 2, so $H$ has two left (or right) cosets. Additionally, this
subgroup itself is abelian. Therefore,
\[
\text{if } g \in H, gH = H = Hg.
\]
But because $H$ has partitioned $G$ into just two cosets, if
\[
    g \notin H \rightarrow gH = G \diagup H \rightarrow Hg = G \diagup H
\]
which means that $H$ is a normal subgroup. That hasn't come up in the text as
of yet, so we're going to pretend that we don't know all about those by now
and discover the property we're going to need: namely, we're just going to
observe that for groups that have the property $gH = Hg$, we may write
$H = gHg^{-1}$. We call this flanking by the coset generator and its inverse
"conjugation". We need a further result that is also not yet in the text:

\textbf{Theorem:} The operation of conjugation preserves the order of the
conjugated element.

\begin{proof}
Suppose $b = gag^{-1}$. Then $b^k = (gag^{-1})^k$. Then
\[
    b^k = (gag^{-1})^k = gag^{-1}gag^{-1}\ldots gag^{-1} 
    = g(aea \ldots eaea)g^{-1} = ga^kg^{-1}.
\]
\end{proof}
As a corollary, if $b$ is order $m$ then by the theorem above,
$e = b^m = ga^mg^{-1} \rightarrow a^m = e$ as those are the only order-4
elements. In the case where $gxg^{-1} = x$, that's an abelian group again and
we've already got that accounted for.

So if $gxg^{=1} = x^3$, that's the same as saying $gxg^{=1} = x^{-1}$. In the
case where $g$ is of order-2, our group $G$ is generated by the following
relations:
\begin{align*}
    gxg^{-1} &= x^{-1}\\
         x^4 &= e\\
         g^2 &=e
\end{align*}
which are the defining properties of the dihedral group of order-8, $D_8$.
These are the symmetries of a square, 2 order-4 operations (corresponding to
rotation clockwise and counter-clockwise by $\frac{\pi}{4})$, an order-1
operation (rotating either way by  $2\pi$), and an order-2 operation (covering
rotation of  $\frac{\pi}{2}$ in either direction). We have also 4 other
order-2 operations, 2 reflections through the corners, and 2 through the sides.

Our last case is the same as the above but we consider all elements in
$G \diagup H$ to be order-4. Then say we take $g \in G \diagup H$: we can see
that $g^2 = x^2$ becuase $g^2$ is order-2 and the only order-2 element in $G$
is $x^2$. We also see that $G$ is non-abelian. If it were, and $gx = xg$, then
\[
    (x^3g)^2 = x^3gx^3g = x^6g^2 = x^2g^2 = x^2x^2 = x^4 = e, 
\]
contradicting the fact that the element $x^3g$ is order-4. Further, we cannot
have $gx = x^2g$ because
\[
gx = x^2g = g^2g = g^3 \rightarrow x = g^2.
\]
$x$ is order-4 and $g^2$ is order-2, so that's just impossible. So we are left
with the realization that $gx = x^3g$. That's enough to give us a Cayley table,
so we may write
\[
 G = \langle x, g\ | x^4 = g^4 = e, x^2 = g^2, gx = x^3g \rangle.
\]

This is the quaternion group, and it is fundamental to particle physics.

\newpage

\textbf{1-5 Conjugate Classes.}

\underline{Problem 1-5.1:} Seperate the elements of $S_5$ into conjugate
classes.

\underline{Solution:} The conjugate classes of $S_5$ are as follows:
\begin{enumerate}
    \item We have the identity cycle, the collection of 1-cycles on 5 elements.
        This has just a single element, $e$.
    \item We have the single 2-cycles, the collection of distinct cycles over
        just two elements. This has ${5 \choose 2} = 10$ elements.
    \item We have the collections of independent double 2-cycles, two 2-cycles
        over independent elements: This class has
        ${5 \choose 2} \cdot {3 \choose 2} \cdot \frac{1}{2} = 15$ elements, where
        %the factor of $\frac{1}{2}$ is to eliminate a double-count because the
        %ordering of each cycle is not important (i.e. $(13)(24) = (24)(13)$).
    \item We have the class of single 3-cycles, which have ${5 \choose 3}
        \cdot2! = 20$ elements (the factorial from the number of internal
        arrangements allowable within a cycle over 3 elements --- $(123) \neq
        (132)$, etc.
    \item We have the collection of independent 3-cycle and 2-cycle
        arrangements, which have ${5 \choose 3} \cdot 2! = 20$ elements. This
        follows exactly the same logic as the previous case, as we have no
        choice in our 2-cycle once the 3-cycle is determined.
    \item We have the class of 4-cycles, which have ${5 \choose 4} \cdot 3! = 30$
        elements.
    \item We have the class of 5-cycles, which simply have $4!=24$ elements.
\end{enumerate}

If we add the number of elements up, we can see that we get $120 = 5!$, which
is exactly what we should expcect from the symmetric group over 5.

\underline{Problem 1-5.2:} Continue the table to $n = 5,6,7$. For $n=5$, give
the cycle structure corresponding to each partition.

\underline{Solution:}
\begin{align*}
    S_5: \text{  }&(5),(41),(32),(31^2),(2^21),(21^3)(1^5)\\
    S_6: \text{  }&(6),(51),(42),(41^2),(3^2),(321),(31^3),(2^3),
                           (2^21^2),(21^4),(1^6)\\
    S_7: \text{  }&(7),(61),(52),(51^2),(43),(421),(41^3),(3^21), (32^2),(321^2),
    (31^4),\\&(2^31), (2^21^3),(21^5),(1^7)
\end{align*}
For $S_5$, the cyclic structures are given by 
\[
    (abcde), (abcd), (abc)(de),(abc), (ab)(cd), (ab) \text{ and } e,
\]
respectively.

\underline{Problem 1-5.3:} Find the number of permutations in each class of
$S_5$.

\underline{Solution:} Already done, see the solution for 1-5.1. Got ahead of
myself! However, I will use the formula for one case to illustrate how it is
done: Say for the partition $(32)$ in the table, we have $n=5$ and so
\[
    \frac{n!}{\prod_i^n \nu_i! i^{\nu_i}} = \frac{120}{(0!(1^0)(4^0)(5^0))
    (1!2^1)(1!3^1)} = \frac{120}{6} = 20,
\]
as there is one 3-cycle and one 2-cycle in this partition, and zero 1-, 4-, and
5-cycles.

\underline{Problem 1-5.4:} By taking successive products of the permutations 
\[
    (1234)(5678) \text{      and       } (1537)(2846)
\]
show that one generates a group of order 8. Seperate the elements of the
group into conjugate classes. Show that this group (the quaternion group) is
isomorphic to the group with elements
\[
1, -1, i, -i, j, -j, k, -k; \\
i^2 = j^2 = k^2 = -1, \text{  } ij = k, jk = i, ki = j.
\]
\underline{Solution:} Let $a = (1234)(5678)$ and $b = (1537)(2846)$. Then from
our previous work discovering the quarternion group, we can see that the group
elements are going to be without doing much work --- though it is
straightforward to verify that this is the group:
\[
G = \langle e, a, a^2, a^3, b, ab, b^3, ba \rangle.
\].

We see a few important relations that will let us define the isomorphism. Those
relations are:
\begin{enumerate}
    \item $a^4 = b^4 = e$,
    \item $a^2 = b^2$,
    \item $a^3 =ab^2 = b^2a$,
    \item $b^3 =ba^2 = a^2b$,
    \item $ba = a^3b$.
\end{enumerate}

So seeing $a^2$ is its own element and $(a^2)^2 = 1$, with 1 already taken, we
assign $a^2 = -1$. Further, we assign $a = i$. Then $a^3 = -i$. We assign
$b = j$, $b^3 = -j$, and $ab = k$, $ba = -k$. This assignment satisfies all the
requirements of the quaternion group and follows the same group relations.

\newpage

\textbf{Invariant subgroups. Factor groups. Homomorphism.}

\underline{Problem 1-6.1:} Without enumerating, find the conjugate subgroups of
$S_3$ in $S_4$; of $S_2$ in $S_4$; of the cyclic group $\{e , (123), 132) \}$
in $S_4$. 

\underline{Solution:} 

\underline{Problem 1-6.2:} Prove that a subgroup of index 2 is necessarily
invariant. 

\underline{Solution:} 

\underline{Problem 1-6.3:} Show that all subgroups of the quarternion group are
invariant.

\underline{Solution:} 

\end{document}
