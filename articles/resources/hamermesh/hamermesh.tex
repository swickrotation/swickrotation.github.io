\documentclass[12pt]{article}
\usepackage{amsmath}
\usepackage{amssymb}
\usepackage{tikz}
\usepackage{parskip}

\title{Solutions to Problems in Morton Hamermesh's ``Group Theory and its
Application to Physical Problems", 2nd Ed. (Dover, 1989)}
\author{William Gertler}

\begin{document}
\maketitle

Something that bothers me about the undergraduate physics curriculum at
Waterloo is that despite the wide array of quantum theory courses you can
(and indeed, must) take, students are not required to take a single course in
abstract algebra. Even those in mathematical physics needn't so much as look at
the pure math calendar to get their degree.

Modern quantum mechanical theory relies very heavily on the
language of groups and algebras \textendash terms that only increase in their importance
the higher up the chain you go. You hear things like ``Poisson bracket" and 
``Lie group" and perhaps even ``Lie Algebra". You hear about Noether's theorem,
but only her work on variational calculus. If you take particle physics,
you may hear about special unitary groups and that sort of thing.

It hardly suffices to take these terms for granted if one wants to dive into the
weeds of modern-day mathematical physics, and there are few places more worthy
of building foundational knowledge than in group theory. So, hopefully you try
the problems out before looking at my solutions here. That's the only way to
really learn, after all. But this will be here should you get truly stuck, and
no mathematician is available to consult.

Hamermesh's text does not number the problems, but the sections are broken up
nicely enough that they're not hard to find, provided I index those sections
here. I will also write out the questions themselves so that searching for them
will be even more simple.

\newpage

\textbf{CHAPTER 1: ELEMENTS OF GROUP THEORY}

\textbf{1-1 Correspondences and Transformations.}

\underline{Problem 1-1.1:} The cross ratio of four points on a line is defined
as
\[ 
\frac{(x_1 - x_2)/(x_3 - x_2)}{(x_1-x_4)/(x_3-x_4)} =
\frac{(x_1 - x_2)(x_3 - x_4)}{(x_1-x_4)(x_3-x_2)},
\]
where $x_1$, $x_2$, $x_3$, $x_4$ are the coordinates of the four points. Show
that the ratio is \textit{invariant} under projective transformation, i.e. that
the cross ratio obtained from the image point has the same form as that for the
object points.

\underline{Solution:} All that needs to be done is verify that the image of x
under a projective transformation results in the same equation as that stated
in the question. The projection of these points gives
\[
    x'_i = \frac{ax_i+b}{cx_i+d}, \text{ where } ad-bc \neq 0.
\]
for $i \in {1,2,3,4}$. By direct calculation, we can see that
\begin{align*}
    & \frac{\left(\frac{ax_1 + b}{cx_1 + d} - \frac{ax_2 + b}{cx_2 + d}\right)
    \left( \frac{ax_3 + b}{cx_3 + d} - \frac{ax_4 + b}{cx_4 + d}\right)}
    {\left(\frac{ax_1 + b}{cx_1 + d} - \frac{ax_4 + b}{cx_4 + d}\right)    
    \left( \frac{ax_3 + b}{cx_3 + d} - \frac{ax_2 + b}{cx_2 + d}\right)}\\
    \\
    &=
    \frac{((cx_2+d)(ax_1+b)-(cx_1+d)(ax_2+b))((cx_4+d)(ax_3+b)-(cx_3+d)(ax_2+b))}
      {((cx_4+d)(ax_1+b)-(cx_1+d)(ax_4+b))((cx_2+d)(ax_3+b)-(cx_3+d)(ax_2+b))}
\end{align*}
after some tedious algebra. After a little more computing, you can simplify
this statement to
\begin{align*}
    & \frac{((ad-bc)(x_1-x_2))((ad-bc)(x_3-x_4))}{((ad-bc)(x_1-x_4))((ad-bc)(x_3-x_2))}\\
    \\
    & = \frac{(x_1 - x_2)(x_3 - x_4)}{(x_1-x_4)(x_3-x_2)}\\
    \\
    & = \frac{(x_1 - x_2)/(x_3 - x_2)}{(x_1-x_4)/(x_3-x_4)}
\end{align*}
which is the same as the cross ratio between object points. Thus, the cross
ratio of a line is an invariant under projective transformations of that line.

\textbf{1-2 Groups. Definition and Examples.}

\underline{Problem 1-2.1:} Show that the cyclic group of order 4 and the Klein
4-group are the only possible structures for groups of order 4.

\underline{Solution:} This problem would be a very quick exercise if at this
point the reader was familiar with Lagrange's theorem, but that hasn't been
covered in the text at this point so I will assume the reader doesn't know this
one. It's still solvable, but it takes more effort (and paper).

Groups need to fulfill certain conditions, which may be reflected in a Cayley
table. In particular, each element in a column (and likewise, a row) may only
appear once per column and per row. Further, each element of the group must
appear in each column and each row. These greatly reduce the number of binary
structures that could represent groups from $4^{16}$ down to 4. We can write
those here:

\begin{tabular}{c | c c c c}
    A & e & a & b & c \\
    \cline{1-5}
    e & e & a & b & c \\
    a & a & e & c & b \\
    b & b & c & a & e \\
    c & c & b & e & a \\
\end{tabular} \quad , \quad 
\begin{tabular}{c | c c c c}
    B & e & a & b & c \\
    \cline{1-5}
    e & e & a & b & c \\
    a & a & b & c & e \\
    b & b & c & e & a \\
    c & c & e & a & b \\
\end{tabular} 

\begin{tabular}{c | c c c c}
    C & e & a & b & c \\
    \cline{1-5}
    e & e & a & b & c \\
    a & a & c & e & b \\
    b & b & e & c & a \\
    c & c & b & a & e \\
\end{tabular} \quad , \quad 
\noindent\begin{tabular}{c | c c c c}
    D & e & a & b & c \\
    \cline{1-5}
    e & e & a & b & c \\
    a & a & e & c & b \\
    b & b & c & e & a \\
    c & c & b & a & e \\
\end{tabular}.

Now, it turns out that the the tables $A$, $B$, $C$ are identical up to an
isomorphism. They are

\begin{align*}
    &\phi_1(A) = \begin{cases}
                \phi(e) &= e\\
                \phi(a) &= b\\
                \phi(b) &= a\\
                \phi(c) &= c\\
                \end{cases} \quad \quad 
    &\phi_2(A) = \begin{cases}
                \phi(e) &= e\\
                \phi(a) &= c\\
                \phi(b) &= b\\
                \phi(c) &= a\\
                \end{cases}
\end{align*}
which take $A \rightarrow B$ and $A \rightarrow C$, respectively. We recognize
these three as one group, which is the cyclic 4-group. Existence and uniqueness
of this group for any positive integer $n$ is clear from modular arithmetic
over the naturals. $D$ is the Klein 4-Group.

\underline{Problem 1-2.2:} Show that the group of order 4 is Abelian.

\underline{Solution:} After those two, this one is very quick. We simply
observe that the Cayley tables for the order 4 groups are all symmetrical.
Thus, the groups they represent are Abelian.

\underline{Problem 1-2.3:} Give a realization of each group of order 4.

\underline{Solution:} An example of the cyclic 4-group is
$\langle \mathbb{Z}_4 , +_4 \rangle$. An example of the Klein 4-group is
$\langle \mathbb{Z}_2 \times \mathbb{Z}_2 , +_2 \rangle $.

\textbf{1-3 Subgroups. Cayley's Theorem.}

\underline{Problem 1-3.1:} Give the elements of the regular subgroup of $S_6$
which is isomorphic with the cyclic group of order 6.

\underline{Solution:} To see the regular subgroup isomorphic to $C_6$, we need
only write one of the cayley tables for that group and read off the results:
\begin{align*}
    \noindent\begin{tabular}{c| c c c c c c}
        & e & a & b & c & d & f\\
       \cline{1-6}
       e & e & a & b & c & d & f\\
       a & a & b & c & d & f & e\\
       b & b & c & d & f & e & a\\
       c & c & d & f & e & a & b\\
       d & d & f & e & a & b & c\\
       f & f & e & a & b & c & d\\ 
    \end{tabular} \rightarrow 
    \begin{cases}
        \binom{123456}{123456} &= e\\
        \binom{123456}{234561} &= (123456)\\
        \binom{123456}{345612} &= (135)(246)\\
        \binom{123456}{456123} &= (14)(25)(36)\\
        \binom{123456}{561234} &= (153)(264)\\
        \binom{123456}{612345} &= (165432)\\
    \end{cases}.
\end{align*}
We can see that this group is indeed regular. These are all the elements, up to
some isomorphisms.

\underline{Problem 1-3.2:} Use Cayley's theorem to find the possible structures
of groups of order 6.

\underline{Solution:} Cayley's theorem states that every group $G$ of order $n$
is isomorphic with a subgroup of the symmetric group $S_n$. In our case, $n=6$.
From our work above, we have already done much of our task. We can further break
down our result: we presume there are no allowed 6-cycles. Then, because the
cycles need be of the same length, we can have two 3-cycles and three 2-cycles,
plus $e$. That gives us
\[ 
\{(123), (132), (12), (13), (23), e\},
\]
which happens to be the symmetric 3-group, $S_3$.

Is that it? It is! How do we know there aren't more? Well, suppose we disallow
3-cycles. Then $G$ has only 2-cycles and the identity. The group has to have 6
elements in total, so we need to have 5 2-cycles as well as the identity. But
by the theorem, that puts us at trying to find a group isomorphic to a subset
of $S_{10}$, not $S_{6}$. It's just too big.

So there are only two ways to structure a group of order 6: the Abelian group
$C_6$ and the non-Abelian group $S_3$.

\textbf{1-4 Cosets. Lagrange's Theorem.}

\underline{Problem 1-4.1:} The cyclic permutations of four symbols form a
subgroup $\mathcal{H}$ of $S_4$. Resolve $S_4$ into left cosets with respect to
$\mathcal{H}$. Compare this resolution with one into right cosets.

\underline{Solution:} Some of the work has been done for us by this point in
the text. We need only look to the previous section to see that the cyclic
group of order 4 is given by

\[ 
C_4 = \{ e, (1234), (13)(24), (1432)\}.
\]



\end{document}

